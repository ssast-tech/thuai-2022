\documentclass{article}

\usepackage{ctex}
\usepackage{xcolor}
\usepackage{hyperref}
\usepackage{dirtree}
\usepackage{gitinfo2}
\usepackage{wasysym}
\usepackage{multicol}
\usepackage{footnote}
\usepackage{listings}
\usepackage{hhline}
\usepackage{siunitx}
\usepackage{fontspec}
\usepackage[margin=1.5cm]{geometry}


\lstset{basicstyle=\ttfamily,breaklines=true}

\title{“巧取智夺”赛道Python SDK}
\author{软院、计算机系联合开发组}
\date{\today\\版本:\gitAbbrevHash}

\begin{document}
\maketitle

\section{简介}

本SDK可以帮助你的AI和评测后端通信。这个SDK由以下文件组成:

\dirtree{%
.1 python.
.2 main.py\DTcomment{选手AI代码文件}.
.2 aisdk\DTcomment{SDK包}.
.3 \_\_init\_\_.py \DTcomment{\texttt{aisdk}包定义}.
.3 entities.py\DTcomment{描述玩家和金蛋状态的类型}.
.3 gamestate.py\DTcomment{和评测逻辑交互的API}.
.3 player\_movement.py\DTcomment{描述玩家移动的类型}.
.3 utils.py\DTcomment{SDK内部使用的和评测逻辑交互的辅助类}.
}

\section{环境配置}

为运行SDK,你需要配置以下环境:
	\begin{itemize}\setlength\itemsep{0em}
		\item \textsc{Python} 3.6及以上版本
		\item 使用 \textsc{Pip} 安装 \texttt{json\_stream\_parser} 包
	\end{itemize}
	对于开发,我们推荐使用 Visual Studio Code 和其 \textsc{Python} 插件的组合。你也可以使用 JetBrains PyCharm 或者其他你喜欢的集成开发环境进行开发。
	在下载SDK后,你可以尝试运行 \texttt{main.py},以检查本地运行环境。如果运行出错,请检查你的本地 \textsc{Python} 环境是否正确配置。

\section{开发}

理论上,你只需要修改\texttt{main.py}这一文件中的 \texttt{update()} 函数。这个函数会在每秒10次\footnote{游戏运行于60fps,每6帧运行一次更新函数,即为每秒运行10次。}的更新中被调用,在其中你可以尝试做出各种动作。请注意:这些函数都不会返回运行的结果,且并不会在调用后立刻体现效果。所有的操作请求都会在 \texttt{update()} 运行结束后一并发送给游戏逻辑。因此,你需要在下一次\texttt{update()}运行时对是否成功执行动作进行检查。

\begin{table}[t]
\caption{SDK提供的数据结构介绍\label{tab:ds}}
\centering
\begin{tabular}{|l|l||l|l|}\hhline{|--||--|}
\multicolumn{2}{|c||}{Player} & \multicolumn{2}{c|}{Egg}\\ \hhline{|--||--|}
\texttt{position} & 玩家坐标 & \texttt{position} & 蛋坐标 \\\hhline{|--||--|}
\texttt{facing} & 表示玩家朝向的单位向量 & \texttt{holder} & 拿蛋玩家,\texttt{None} 表示放在地上\\[0pt]\hhline{|-|-||--|}
\texttt{status} & 玩家运动状态 & \texttt{score} & 蛋的分数 \\\hhline{|-|-|:==:}
\texttt{endurance} & 玩家的耐力值,$-1$代表逻辑版本需更新 & \multicolumn{2}{c|}{} \\\hhline{|-|-|:==:}
\texttt{holding} & 玩家拿的蛋,\texttt{None}表示空手 &\multicolumn{2}{c|}{PlayerMovement}  \\ \hhline{:==:|--|}
\multicolumn{2}{|c||}{Team} &\texttt{STOPPED} & 玩家停在原地 \\ \hhline{|--||--|} 
\texttt{RED} & 红队 & \texttt{WALKING} & 玩家正在走路  \\ \hhline{|--||--|}
\texttt{YELLOW} & 黄队 &\texttt{RUNNING} & 玩家正在跑步  \\ \hhline{|--||--|}
\texttt{BLUE} & 蓝队 & \texttt{SLIPPED} & 玩家因碰撞滑倒,本回合操作无效\\ \hhline{|--||--|}\end{tabular}
\end{table}

SDK中提供的主要数据结构见表\ref{tab:ds}。

\subsection{接口}

所有公共接口均位于 \texttt{gamestate.py} 中。根据 \textsc{Python} 的模块导入,模块本身即为单例模式。代码中要使用相应接口,只需要导入 \texttt{aisdk.gamestate} 这一模块即可。

\begin{description}
\item[玩家控制] 通过对玩家对象属性的读取和赋值,以尝试获取和修改玩家的具体信息。下文中 \lstinline{p} 代表玩家对象。

注意:在对玩家代理对象赋值操作后,立刻读取得到的仍然是原来的值!这是因为修改状态的操作尚未被评测端接受,所有修改操作会在更新回调结束后一并发送给评测端。你应当在下次调用更新回调函数时加以检查。
\begin{itemize}
	\item
  \lstinline{Player(player_id: int)}\\
  \lstinline{Player.get_player_by_team_and_id(team_id: Team, player_id_in_team: int)}\\[-2pt]
获得 Player 对象。若设总的编号为 $x$,则队伍$t$和队内编号$y$由以下公式得出:
\[t = x \div 4, y = x \bmod 4\]
其中 $t=0,1,2$分别对应红、黄、蓝队。

\item \lstinline{p.player_id}\\[-2pt]
只读。玩家的 id,范围为 $0 \sim 11$.

\item \lstinline{p.position}\\[-2pt]
只读。玩家的坐标。

\item \lstinline{p.endurance}\\[-2pt]
只读。玩家的耐力值。若为 $-1$ 说明读取失败,请更新游戏逻辑。

\item \lstinline{p.team, p.id_on_team}\\[-2pt]
只读。获得这个玩家所在队伍和队内编号,即上文的 $t, y$。

\item \lstinline{p.holding}\\[-2pt]
只读。玩家拿的蛋对应的 \lstinline{Egg} 对象,若为拿蛋即为 \lstinline{None}.

\item
\lstinline{p.status}\\[-2pt]
通过对玩家对象\lstinline{p} 的 \lstinline{status} 属性进行赋值,以尝试设置移动状态。
赋的新值必须为 \lstinline{PlayerMovement} 类型;尝试改变不在当前AI队伍的玩家的状态会导致抛出异常。
如果不满足条件,则设置失败。具体失败的情形为:
\begin{itemize}\setlength\itemsep{0em}
\item 该玩家已经摔倒:此时在站起来(恢复成静止)前不能进行任何操作
\item 抱着蛋时尝试奔跑
\item 体力值不够时尝试奔跑
\end{itemize}

\item
\lstinline{p.facing}\\[-2pt]
直接读取值即为玩家当前朝向。通过给玩家对象 \lstinline{p} 的 \lstinline{facing} 属性赋值,设置其朝向(用于走路、奔跑)。注意:若赋值的是非单位向量,则会将其变为同向单位向量。传入零向量或者模长过小的向量时,评测逻辑行为未定义。

\end{itemize}

\item[金蛋基本信息] 通过对 \lstinline{Egg} 对象属性的读取,以尝试获取金蛋的具体信息。下文中 \lstinline{e} 代表金蛋对象。

\begin{itemize}
\item
\lstinline{Egg(egg_id: int)}\\[-2pt]
传入金蛋编号,获得其基本信息对象。

\item \lstinline{e.egg_id}\\[-2pt]
只读。金蛋 \lstinline{e} 的id.

\item \lstinline{e.position}\\[-2pt]
只读。金蛋的坐标。

\item \lstinline{e.holder}\\[-2pt]
只读。拿着这个金蛋的玩家对象,若为 \lstinline{None} 表示蛋在地上。

\item \lstinline{e.score}\\[-2pt]
只读。金蛋的分数。
\end{itemize}

\item[金蛋控制] 下文中 \lstinline{p} 代表玩家对象。

\begin{itemize}

\item
\lstinline{p.try_grab_egg(egg_id: int)}\\[-2pt]
让当前AI队伍中某玩家对象 \lstinline{p} 尝试抓取金蛋。只有满足下列条件时,抓取才能成功:
\begin{itemize}\setlength\itemsep{0em}
\item 蛋在地上且糖豆人中心和蛋表面距离不超过 \SI{0.1}{\meter}(即到蛋中心距离不超过\SI{0.69}{\meter})\footnote{$\diameter_{\text{玩家}}=\SI{0.48}{\meter},\diameter_{\text{金蛋}}=\SI{0.7}{\meter}$}
\item 该蛋由别人拿取,且玩家和蛋距离同样不超过\SI{0.69}{\meter}
\item 多人在同回合抢同一个蛋时,某人和蛋距离最近
\end{itemize}

\item
\lstinline{p.try_drop_egg(radian: float)}\\[-2pt]
让当前AI队伍中某玩家对象 \lstinline{p} 尝试放置金蛋。参数中的弧度为以$+x$轴为极轴的极坐标系下,放置蛋相对玩家的方位。蛋在放置后会和玩家刚好相切。
只有满足下列条件时,放置才能成功:
	\begin{itemize}\setlength\itemsep{0em}
	\item 该玩家手中有蛋
	\item 蛋放下后不会卡在他人或其他蛋碰撞箱内
	\item 蛋放下后不会卡在墙内
	\end{itemize}
\end{itemize}

\end{description}

\subsection{上交代码}

按照Saiblo的要求,提交 Python 语言代码只需要打包上传SDK文件夹下的所有文件即可。注意上传文件中,\texttt{main.py} 必须位于压缩包的顶层文件夹。

\end{document}
